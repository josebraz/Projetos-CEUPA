\documentclass{article}

\usepackage{geometry}
\usepackage[utf8]{inputenc}
\usepackage{amsmath, amsfonts, amssymb}
\begin{document}
	\title{Projeto OH \footnote{OH significa Organização e Higiene}}
	\author{Jose Braz}
	\date{26/08/2018}

	\maketitle

	\section{Resumo}

	Uso de Programação linear para resolver o problema da melhor alocação de
	moradores para a limpeza das áreas do andar superior da Casa III da instituição
	CEUPA (Casa Estudantil Universitária de Porto Alegre). O problema consiste
	em escolher os moradores para cada área de forma que todas as áreas sejam
	distribuídas o mais uniformemente possível (com a menor variância). É
	preciso garantir que todas as áreas sejam limpas duas vezes na semana e todos
	os moradores devem limpar exatamente uma área na semana.
	A distribuição se dá em duas etapas:

	\begin{enumerate}
		\item Uma fila circular de moradores para as áreas B, E e F
		\item Uma alocação para as restantes áreas comuns A, C e D
	\end{enumerate}

	O foco será na segunda etapa.

	\section{Modelo}

	Objective \\
	\begin{equation} \label{obj}
		Max \sum_{k \in C}{\sum_{i \in CA}{P_{QAM_k, i} * M_{i,k}}}
	\end{equation}

	Subject to \\
	% Restrição de linha
	\begin{equation} \label{row_r}
		\sum_{k \in C}{M_{i,k}} = \sum_{k \in C}{QAM_{i,k}} + 2 \quad \forall i \in CA
	\end{equation}

	% Restrição de coluna
	\begin{equation} \label{columm_r}
		\sum_{i \in CA}{M_{i,k}} = \sum_{i \in CA}{QAM_{i,k}} + 1 \quad \forall k \in C
	\end{equation}

	% Restrição de quantidade
	\begin{equation} \label{quant_r}
		M_{i,k} \geq QAM_{i,k} \quad \forall k \in C \quad \forall i \in CA
	\end{equation}

	Onde:
	\begin{itemize}
		\item $CA$ é o conjuto das áreas comuns ($\{A, C, D\}$)
		\item $C$ é o conjuto dos CLEANERS (moradores que irão limpar as áreas comuns)
		\item $QAM_{i,k}$ onde: $i \in CA$ e $k \in C$ \\
			  representa a quantidade de vezes que o morador $k$ fez a área $i$
			  historicamente (guarda a informação para as proximas computações)
		\item $M_{i,k}$ onde: $i \in CA$ e $k \in C$ \\
			  representa a quantidade de vezes que o morador $k$ fez a área $i$
			  mais a computação dos próximos a limpar as áreas.
		\item $P_{\bar{x},y}$ é a função que retorna um inteiro correspondente ao peso
			  que a área $y$ tem de acordo com a diferença com o maior elemento
			  do vetor de áreas comuns $\bar{x}$. A função é dada por:
			  $$P_{\bar{x},y} = 5^{maximun(\bar{x}) - \bar{x}_y}$$
	\end{itemize}
	O objetivo (\ref{obj}) é um somatório ponderado de acordo com a função
	$P_{\bar{x},y}$, logo quanto maior o peso, maior será a prioridade em
	aumentar o $M_{i,k}$, fazendo com que a diferença entre as áreas $i$ feitas
	por $k$ seja pequena. A restrição de linha (\ref{row_r}) garante
	que cada área será feita duas vezes naquela semana. A restrição de coluna
	(\ref{columm_r}) garante que cada morador irá fazer a limpeza exatamente
	uma vez naquela semana. A restrição de quantidade (\ref{quant_r}) garante
	que a matriz $M$ será maior que a matriz histórica $QAM$, ou seja, conserva
	a informação das limpezas feitas nas semanas anteriores. A diferença entre
	$M_{i,k}$ e $QAM_{i,k}$ gera uma matriz binária que representa qual morador
	deve limpar determinada área naquela semana.

	\section{Exemplo}
	O algorítmo aplicado para o seguinte QAM \\

	\begin{tabular}{c|cccccc}
		&	Lais &	Gustavo &	Jose &	Filipe &	Renata &	Tales \\ \hline
		A &	5	 &  3 		&	3	 & 	5      &  	4	   &	5 \\
		C &	6	 &  3 		&	4	 & 	5      &  	5	   &	3 \\
		D &	7	 &  5 		&	3	 & 	3      &  	6	   &	4 \\
	\end{tabular}
	\vspace{1cm}

	Objetivo:
	\begin{multline}
		Max \quad 25 M_{A,Lais} + 5 M_{C,Lais} + M_{D,Lais} + 25 M_{A,Gustavo} + \\
		25 M_{C,Gustavo} + M_{D,Gustavo} + 5 M_{A,Jose} + M_{C,Jose} + 5 M_{D,Jose} + \\
		M_{A,Filipe} + M_{C,Filipe} + 25 M_{D,Filipe} + 25 M_{A,Renata} + \\
		5 M_{C,Renata} + M_{D,Renata} + M_{A,Tales} + 25 M_{C,Tales} + 5 M_{D,Tales}
	\end{multline}
	\vspace{5mm}

	Restrições de Linha: \\
	$M_{A,Lais} + M_{A,Gustavo} + M_{A,Jose} + M_{A,Filipe} + M_{A,Renata} + M_{A,Tales} = 27$ \\
    $M_{C,Lais} + M_{C,Gustavo} + M_{C,Jose} + M_{C,Filipe} + M_{C,Renata} + M_{C,Tales} = 28$ \\
    $M_{D,Lais} + M_{D,Gustavo} + M_{D,Jose} + M_{D,Filipe} + M_{D,Renata} + M_{D,Tales} = 30$
	\vspace{5mm}

	Restrições de Coluna: \\
    $M_{A,Lais} + M_{C,Lais} + M_{D,Lais} = 19$ \\
    $M_{A,Gustavo} + M_{C,Gustavo} + M_{D,Gustavo} = 12$ \\
    $M_{A,Jose} + M_{C,Jose} + M_{D,Jose} = 11$ \\
    $M_{A,Filipe} + M_{C,Filipe} + M_{D,Filipe} = 14$ \\
    $M_{A,Renata} + M_{C,Renata} + M_{D,Renata} = 16$ \\
    $M_{A,Tales} + M_{C,Tales} + M_{D,Tales} = 13$
	\vspace{5mm}

	Restrições de Quantidade: \\
    $M_{A,Lais} \geq 5$ \\
    $M_{C,Lais} \geq 6$ \\
    $M_{D,Lais} \geq 7$ \\
    $M_{A,Gustavo} \geq 3$ \\
    $M_{C,Gustavo} \geq 3$ \\
    $M_{D,Gustavo} \geq 5$ \\
    $M_{A,Jose} \geq 3$ \\
    $M_{C,Jose} \geq 4$ \\
    $M_{D,Jose} \geq 3$ \\
    $M_{A,Filipe} \geq 5$ \\
    $M_{C,Filipe} \geq 5$ \\
    $M_{D,Filipe} \geq 3$ \\
    $M_{A,Renata} \geq 4$ \\
    $M_{C,Renata} \geq 5$ \\
    $M_{D,Renata} \geq 6$ \\
    $M_{A,Tales} \geq 5$ \\
    $M_{C,Tales} \geq 3$ \\
    $M_{D,Tales} \geq 4$
	\vspace{5mm}

	Variável M \\
    $M_{i,k} \geq 0, integer, \forall i \in \{A,C,D\}, k \in \{Lais,Gustavo,Jose,Filipe,Renata,Tales\}$ \\

	O resultado da diferença entre $M$ e $QAM$ foi: \\
	
	\begin{tabular}{c|cccccc}
		&	Lais &	Gustavo &	Jose &	Filipe &	Renata &	Tales \\ \hline
		A &	1	 &  0 		&	0	 & 	0      &  	1	   &	0 \\
		C &	0	 &  1 		&	0	 & 	0      &  	0	   &	1 \\
		D &	0	 &  0 		&	1	 & 	1      &  	0	   &	0 \\
	\end{tabular}


\end{document}
