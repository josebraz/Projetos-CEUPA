\documentclass{article}

\usepackage{geometry}
\usepackage[utf8]{inputenc}
\usepackage{amsmath, amsfonts, amssymb}
\begin{document}
    \title{Projeto OH \footnote{OH significa Organização e Higiene}}
    \author{Jose Braz}
    \date{26/08/2018}

    \maketitle

    \section{Resumo}

    Uso de Programação linear para resolver o problema da melhor alocação de
    moradores para a limpeza das áreas do andar superior da Casa III da instituição
    CEUPA (Casa Estudantil Universitária de Porto Alegre). As áreas são 
        divisões da casa nomeadas com uma letra e divididas em dois grupos:
        \begin{itemize}
            \item Áreas comuns. $C = {A, C, D}$
            \item Áreas especias. $E = {B, E, F}$
        \end{itemize}
    O problema consiste em escolher os moradores para cada área de forma 
    que todas as áreas sejam
    distribuídas o mais uniformemente possível, ou seja, não deve haver
    moradores que fizeram muito uma determinada área, mas pouco outra. É
    preciso garantir, ainda, que todas as áreas sejam limpas duas vezes na 
    semana e todos os moradores devem limpar exatamente uma área na semana 
    (com excessão da Área F). 
    A distribuição das áreas se dá em duas etapas:

    \begin{enumerate}
        \item Uma fila circular de moradores para as áreas B, E e F
        \item Uma alocação para as restantes áreas comuns A, C e D
    \end{enumerate}

    O foco será na segunda etapa. Uma vez que a primeira é só a aplicação
        de um estrutura de dados simples.

    \section{Modelo}

    Objective \\
    \begin{equation} \label{obj}
        Max \sum_{j \in L}{\sum_{i \in C}{P(Q_j, i) * M_{i,j}}}
    \end{equation}

    Subject to \\
    % Restrição de linha
    \begin{equation} \label{row_r}
        \sum_{j \in L}{M_{i,j}} = \sum_{j \in L}{Q_{i,j}} + 2 \quad \forall i \in C
    \end{equation}

    % Restrição de coluna
    \begin{equation} \label{columm_r}
        \sum_{i \in C}{M_{i,j}} = \sum_{i \in C}{Q_{i,j}} + 1 \quad \forall j \in L
    \end{equation}

    % Restrição de quantidade
    \begin{equation} \label{quant_r}
        M_{i,j} \geq Q_{i,j} \quad \forall j \in L \quad \forall i \in C
    \end{equation}

    Onde:
    \begin{itemize}
        \item $C$ é o conjuto das áreas comuns ($\{A, C, D\}$)
        \item $L$ é o conjuto dos moradores que irão limpar as áreas comuns
              naquela semana. 
        \item $Q_{i,j}$ onde: $i \in C$ e $j \in L$ \\
              representa a quantidade de vezes que o morador $j$ fez 
              a área $i$ historicamente (guarda a informação para as 
              proximas computações). Note que o $Q_j$ representa quantas
              vezes o morador $j$ limpou as áreas do conjunto $C$
        \item $M_{i,j}$ onde: $i \in C$ e $j \in L$ \\
              representa a quantidade de vezes que o morador $j$ fez a área $i$
              mais a computação dos próximos a limpar as áreas. Ou seja, $M$ contém
              os valores de $Q$ e mais os valores que a computação desse modelo
              irá retornar. $M$ é a variável em uma implementação computacional.
        \item $P(\bar{x},a)$ onde: $\bar{x} \in \mathbb{N}$ e $a \in C$ mapeada por:
              $$i(a)= 
              \begin{cases}
                    1, & \text{se } a = A\\
                    2, & \text{se } a = C\\
                    3, & \text{se } a = D\\
              \end{cases}
              $$
              $P$ é a função que retorna um valor correspondente ao "peso"
              que a área $a$ tem, calculado de acordo com a diferença com o 
              maior elemento do vetor de áreas comuns $\bar{x}$. A função é dada por:
              $$P(\bar{x},a) = 5^{maximun(\bar{x}) - \bar{x}_{i(a)}}$$
              Note que, quanto menor o valor de $\bar{x}_{i(a)}$ maior será o resultado
              da função e, como o objetivo é maximizar a equação (\ref{obj}), 
              priorizará a escolha dessa área $a$ em detrimento das outras.
    \end{itemize}
    
    O objetivo (\ref{obj}) é um somatório ponderado de acordo com a função
    $P_{\bar{x},y}$, logo quanto maior o peso, maior será a prioridade em
    aumentar o $M_{i,j}$, fazendo com que a diferença entre as áreas $i$ feitas
    pelo morador $j$ seja pequena. A restrição de linha (\ref{row_r}) garante
    que cada área será feita duas vezes naquela semana. A restrição de coluna
    (\ref{columm_r}) garante que cada morador irá fazer a limpeza exatamente
    uma vez naquela semana. A restrição de quantidade (\ref{quant_r}) garante
    que a matriz $M$ será maior que a matriz histórica $Q$, ou seja, conserva
    a informação das limpezas feitas nas semanas anteriores. A diferença entre
    $M_{i,j}$ e $Q_{i,j}$ gera uma matriz binária que representa qual morador
    deve limpar determinada área naquela semana.

    \section{Exemplo}
    Supondo um $Q$ como segue, a computação deve ser capaz de determinar a melhor
    alocação dessas pessoas para as áreas, de acordo com os critérios mencionados. \\

    \begin{tabular}{c|cccccc}
          &    Lais  &    Gustavo &    Jose  &     Filipe &      Renata  &    Tales \\ 
        \hline
        A &    5     &  3         &    3     &     5      &      4       &    5     \\
        C &    6     &  3         &    4     &     5      &      5       &    3     \\
        D &    7     &  5         &    3     &     3      &      6       &    4     \\
    \end{tabular}
    \vspace{1cm}

    Objetivo:
    \begin{multline}
        Max \quad 25 M_{A,Lais} + 5 M_{C,Lais} + M_{D,Lais} + 25 M_{A,Gustavo} + \\
        25 M_{C,Gustavo} + M_{D,Gustavo} + 5 M_{A,Jose} + M_{C,Jose} + 5 M_{D,Jose} + \\
        M_{A,Filipe} + M_{C,Filipe} + 25 M_{D,Filipe} + 25 M_{A,Renata} + \\
        5 M_{C,Renata} + M_{D,Renata} + M_{A,Tales} + 25 M_{C,Tales} + 5 M_{D,Tales}
    \end{multline}
    \vspace{5mm}

    Restrições de Linha: \\
    $M_{A,Lais} + M_{A,Gustavo} + M_{A,Jose} + M_{A,Filipe} + M_{A,Renata} + M_{A,Tales} = 27$ \\
    $M_{C,Lais} + M_{C,Gustavo} + M_{C,Jose} + M_{C,Filipe} + M_{C,Renata} + M_{C,Tales} = 28$ \\
    $M_{D,Lais} + M_{D,Gustavo} + M_{D,Jose} + M_{D,Filipe} + M_{D,Renata} + M_{D,Tales} = 30$
    \vspace{5mm}

    Restrições de Coluna: \\
    $M_{A,Lais} + M_{C,Lais} + M_{D,Lais} = 19$ \\
    $M_{A,Gustavo} + M_{C,Gustavo} + M_{D,Gustavo} = 12$ \\
    $M_{A,Jose} + M_{C,Jose} + M_{D,Jose} = 11$ \\
    $M_{A,Filipe} + M_{C,Filipe} + M_{D,Filipe} = 14$ \\
    $M_{A,Renata} + M_{C,Renata} + M_{D,Renata} = 16$ \\
    $M_{A,Tales} + M_{C,Tales} + M_{D,Tales} = 13$
    \vspace{5mm}

    Restrições de Quantidade: \\
    $M_{A,Lais} \geq 5$ \\
    $M_{C,Lais} \geq 6$ \\
    $M_{D,Lais} \geq 7$ \\
    $M_{A,Gustavo} \geq 3$ \\
    $M_{C,Gustavo} \geq 3$ \\
    $M_{D,Gustavo} \geq 5$ \\
    $M_{A,Jose} \geq 3$ \\
    $M_{C,Jose} \geq 4$ \\
    $M_{D,Jose} \geq 3$ \\
    $M_{A,Filipe} \geq 5$ \\
    $M_{C,Filipe} \geq 5$ \\
    $M_{D,Filipe} \geq 3$ \\
    $M_{A,Renata} \geq 4$ \\
    $M_{C,Renata} \geq 5$ \\
    $M_{D,Renata} \geq 6$ \\
    $M_{A,Tales} \geq 5$ \\
    $M_{C,Tales} \geq 3$ \\
    $M_{D,Tales} \geq 4$
    \vspace{5mm}

    Variável M \\
    $M_{i,j} \geq 0, integer, \forall i \in \{A,C,D\}, j \in \{Lais,Gustavo,Jose,Filipe,Renata,Tales\}$ \\

    O resultado da diferença entre $M$ e $Q$ foi: \\
    
    \begin{tabular}{c|cccccc}
          &    Lais  &  Gustavo   &    Jose  &     Filipe &      Renata  &    Tales \\ \hline
        A &    1     &  0         &    0     &     0      &      1       &    0 \\
        C &    0     &  1         &    0     &     0      &      0       &    1 \\
        D &    0     &  0         &    1     &     1      &      0       &    0 \\
    \end{tabular}
    
    \vspace{1cm}
    
    Analizando os resultamos vemos que:
    \begin{itemize}
        \item Lais e Renata devem limpar a área A na semana em questão
        \item Gustado e Tales devem limpar a área C na semana em questão
        \item Jose e Filipe devem limpar a área D na semana em questão
    \end{itemize}
\end{document}
